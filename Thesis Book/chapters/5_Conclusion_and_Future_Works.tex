\documentclass[document.tex]{subfiles}
\begin{document}

\chapter{Conclusion and Future Works}

\section{Conclusion}
\noindent The proposed target class oriented subspace detection (TCOSD) provide improved subspace for the task of classification. The classification performance of the proposed method is justified using KSVM. It can be seen that the proposed method is able to detect a better subspace which can provide the best classification accuracy among the standard approaches studied. This is because the proposed method selects the most relevant subset of images for required classes. It also consider the spatial information while performing the feature selection which was absent in PCA only method. Similarly, PCA+NMI considers all the classes together and come up with suboptimal solution. Therefore, the proposed method is most suitable for ground object detection and its classification.
\section{Limitations}
\noindent The proposed method needs a huge amount of computational time and processing power. This approach may become very costly when the number of training and testing samples and number of class labels are very large.
\section{Future Works}
The proposed method (TCOSD) needs some further improvement to handle the complex class relationships where only a few feature may not capable to complete the task.
\end{document}